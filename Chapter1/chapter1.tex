%%%%%%%%%%%%%%%%%%%%%%%%%%%%%%%%%%%%%%%%%%%%%%%%%%%%%%%%%%%%%%%%%%%%%%%%%%%%%%%%%%%%%%%%%%%
%                                     MSthesis v1.1                                       %
%                            By Quan Zou <quan.zou@gmail.com>                             %
%                            Version 1.1 released 01/10/2010                              %
%%%%%%%%%%%%%%%%%%%%%%%%%%%%%%%%%%%%%%%%%%%%%%%%%%%%%%%%%%%%%%%%%%%%%%%%%%%%%%%%%%%%%%%%%%%
%
%------------------------------------------------------------------------------------------
% thesis chapter
%------------------------------------------------------------------------------------------

\chapter{Introduction}\label{chap1}
\ifpdf
    \graphicspath{{Chapter1/Chapter1Figs/PNG/}{Chapter1/Chapter1Figs/PDF/}{Chapter1/Chapter1Figs/}}
\else
    \graphicspath{{Chapter1/Chapter1Figs/EPS/}{Chapter1/Chapter1Figs/}}
\fi

\section{Background of synapse transmission}

% (chemical) synaptic transmission in the brain
The synapse is the functional communication unit between one neuron
and another.  It is the target at which information transfer happens.
\fref{Syn_Intro} schematizes a synapse illustrating the pre-synaptic
bouton at the end of the axon, and a spine on the dendrite. The
terminology of pre-synaptic and post-synaptic defines the direction
of signal flow. Following an electrical signal neurotransmitter is
secreted from the pre-synaptic nerve terminal of a chemical synapse. 
The neurotransmitter binds to post-synaptic receptors to mediate flux
of ions across the membrane, driving the membrane potential away from
resting voltage and causing a so called post-synaptic potential (PSP).
Synaptic potential may trigger an action potential in the cell body,
which runs down the axon to be conveyed from the nerve terminal onto
the next neuron~\cite{Ral98}.  The two neuron are not directly
connected but communicate via the release of transmitters in the
synaptic cleft.\\

% electrical synapse
Most synapses are chemical: neurotransmitters are released from one
neuron and bind to specific receptors on another neuron, generating
an electrical signal in the latter neuron. In addition to chemical
synapses, there are electrical synapses with specialized channels
(gap junctions) that allow current to flow directly from one neuron
to another.  \\

% postsynaptic potentials
The number of synaptic connections a neuron forms can be
extraordinarily large and many afferents can interact and influence a
post-synaptic neuron, by either excitatory or inhibitory effects,
depending on the ions that permeate the channels on the postsynaptic
side operated by the receptor. These can be {\epsp}s (EPSPs) or
{\ipsp}s (IPSPs). These potentials can vary in size. An IPSP pushes
the membrane potential down to more negative values and away from its
firing potential. An EPSP moves the potential in a positive direction,
toward firing threshold. Finally, if a large EPSP reaches the firing
threshold, the neuron fires a spike (action potential, AP). This
spike can be considered as all-or-none, because it is of stereotyped
shape and amplitude, and is not graded. The action potential is the
signal that can be sent down the axon to create a PSP in another
neuron~\cite{Lyn02}.\\

\nomenclature[zipsp]{$IPSP$}{\ipsp}				% first letter z is for Acronyms
\nomenclature[zepsp]{$EPSP$}{\epsp}				% first letter z is for Acronyms

% Biophysics of glutamatergic synapse
Most excitatory contacts between neurons in cerebral cortex release
glutamate as their transmitter. In most parts of this thesis, we
consider the generic form of glutamatergic synapse
(\fref{Syn_Intro}), which is found throughout the cortex as well as
in many other regions of the central nervous system.  Upon arrival of
a post-synaptic AP, {\ca} enters the post-synaptic terminal through
specialized voltage-gated calcium channels. The local intense rise of
{\ca} concentration triggers the fusion of docked vesicles with the
cell membrane and the release of their contents in the synaptic 
cleft~\cite{Edw95}.  The fusion of one vesicle releases about 
1000 to 4000 transmitter molecules in less than a millisecond, 
which activate glutamatergic postsynaptic receptors that mediate 
{\epsp} (EPSC) recorded on the postsynaptic side~\cite{Clm96,
Edw95}. After release transmitter molecules are rapidly removed 
from the cleft either by diffusion or binding to glutamate
transporters~\cite{Dim97, Rsk98}. \\

%\InsertFig{\IncludeGraphicsW{fig_name}{4in}}
%{Title}{Label}
%{Captions}

\InsertFig{\IncludeGraphicsW{syn_intro}{5.5in}{88 350 532 775}}
{Basic synapse design}{Syn_Intro}
{Electrical signals in the pre-synaptic neuron cause the vesicles 
(big dots) to release neurotransmitters (little dots) into the 
synaptic cleft. They diffuse across the cleft and bind the surface 
receptors, which then triggers an influx of ions into the 
post-synaptic neuron causing the synaptic potential. (modified from 
Lytton, 2002~\cite{Lyn02}.)}

% 2 type of Glu-receptors
The release of glutamate activates two different types of
ligand-gated ion channels, AMPA (\ampa)/kainate receptors and NMDA
(\nmda) receptors.  AMPA receptors activate quickly in less than a
millisecond and mediate the major contribution of glutamate activated
excitatory transmission.  The activation of NMDA receptors occurs on
a much slower time-scale of tens of milliseconds.  Around resting
potential ($\sim -$70~mV), NMDA receptors are nonconducting due to a
voltage dependent magnesium block but only activate both glutamate
release is combined with postsynaptic depolarization~\cite{Kzm97}.
The number of AMPA receptors within a glutamatergic synapse has been
estimated as 30 to 100 receptors, whereas there are only few, i.e. 
less than ten NMDA receptors~\cite{Edw95, Spt95}. The latter hardly
contribute to fast excitatory transmission and phenomena of
short-term plasticity, but may be important for development and
long-term changes in synaptic efficacy~\cite{Kzm97, Mar98}.  Although
a variety of electrophysiological and anatomical studies have been
performed on glutamatergic synapses~(reviewed in~\cite{Edw95a,
Wls98}), essential steps of the transmission process are not
understood in detail. \\

\nomenclature[gdelta]{$\delta(t)$}{synaptic pulse or neuronal spike}			% first letter g is for Greek

%------------------------------------------------------------------------------------------

%\InsertFig{\IncludeGraphicsW{fig_name}{4in}}
%{Title}{Label}
%{Captions
%}


%%% Local Variables: 
%%% mode: latex
%%% TeX-master: "../thesis"
%%% End: 
