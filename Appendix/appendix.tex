%%%%%%%%%%%%%%%%%%%%%%%%%%%%%%%%%%%%%%%%%%%%%%%%%%%%%%%%%%%%%%%%%%%%%%%%%%%%%%%%%%%%%%%%%%%
%                                     MSthesis v1.1                                       %
%                            By Quan Zou <quan.zou@gmail.com>                             %
%                            Version 1.1 released 01/10/2010                              %
%%%%%%%%%%%%%%%%%%%%%%%%%%%%%%%%%%%%%%%%%%%%%%%%%%%%%%%%%%%%%%%%%%%%%%%%%%%%%%%%%%%%%%%%%%%
%
%------------------------------------------------------------------------------------------
% thesis appendix
%------------------------------------------------------------------------------------------
\chapter{} \label{Appdx}

\ifpdf
    \graphicspath{{Appendix/AppendixFigs/PNG/}{Appendix/AppendixFigs/PDF/}{Appendix/AppendixFigs/}}
\else
    \graphicspath{{Appendix/AppendixFigs/EPS/}{Appendix/AppendixFigs/}}
\fi

\section{Example of an equation} \label{Appdx1}

The model equations implemented in both numeric simulation and analog circuits were as follows:

\bea
\label{AppdxB_1}
C_{MEM}\frac{dV_{MEM}}{dt} &\ =\ & -\sum( I_{Ion} ) + \sum(I_{Ext})	~ ,
\eea

where $I_{\mathrm {Ion}}$ are the ionic currents, $I_{\mathrm {Ext}}$ the synaptic and stimulation currents. 
These currents were described by:

\bea
\label{AppdxB_2}
I_{Ion} &\ =\ & {\bar g}_{Ion} m^ph^q(V_{MEM}-V_{EQ}) ~ , \\
{\tau}_{m}(V_{MEM})\frac{dm}{dt} = m_{\infty}(V_{MEM}) - m ~ , &&
m_{\infty}(V_{MEM}) =
\frac{1}{1+exp\Big(-\frac{V_{MEM}-V_{OFFSETm}}{V_{SLOPEm}}\Big)} ~ , \nonumber \\
{\tau}_{h}(V_{MEM})\frac{dh}{dt} = h_{\infty}(V_{MEM}) - h ~ , &&
h_{\infty}(V_{MEM}) =
\frac{1}{1+exp\Big(-\frac{V_{MEM}-V_{OFFSETh}}{V_{SLOPEh}}\Big)} ~,
\nonumber
\eea

Due to electronics technical constraints, the area of the integrated cells is 0.00022~{\cmcm}, %$cm^2$, 
which corresponds a value for the membrane capacitance: C$_{MEM}$ $=$ 1~{\mufcm2}.\\


\section{Example of a table} \label{Appdx2}

The Memory map\footnote{provided by Dr. Yanick Bornat, IXL, Universit\'e
Bordeaux 1, Bordeaux, France} of PAX I was given as the following, through
which communication between the digital hardware and the software layer
is maintained~\cite{Zou09}. \\

{\noindent
\singlespacing
\begin{tabular}{|p{2cm}|p{2cm}|p{.8cm}|p{.8cm}|p{.8cm}|p{.8cm}|p{.8cm}|p{.8cm}|p{.8cm}|p{.8cm}|}
\hline \multicolumn{2}{|c|}{Address (bytes)} & \multicolumn{8}{|c|}{Memory (bits)}\\
\hline  DEC   & HEX & 0000 & 0000 & 0000 & 0000 & 0000 & 0000 & 0000 & 0000 \\
%\hline DEC &   HEX & \multicolumn{2}{|c|}{+03} & \multicolumn{2}{|c|}{+02} & \multicolumn{2}{|c|}{+01} & \multicolumn{2}{|c|}{+00} \\
\hline   0 & 0x000 & \multicolumn{8}{|l|}{Identification Number} \\
\hline   4 & 0x004 & \multicolumn{4}{|l|}{State Registers} & \multicolumn{4}{|l|}{Led Status} \\
\hline   8 & 0x008 & \multicolumn{8}{|l|}{System Clock Value} \\
\hline  12 & 0x00C & \multicolumn{8}{|l|}{System Clock cycle duration} \\
\hline  16 & 0x010 & \multicolumn{8}{|>{\columncolor[rgb]{0.82,0.82,0.82}}l|}{Reserved} \\
\hline  20 & 0x014 & \multicolumn{8}{|>{\columncolor[rgb]{0.82,0.82,0.82}}l|}{Reserved} \\
\hline  24 & 0x018 & \multicolumn{8}{|>{\columncolor[rgb]{0.82,0.82,0.82}}l|}{Reserved} \\
\hline  28 & 0x01C & \multicolumn{8}{|>{\columncolor[rgb]{0.82,0.82,0.82}}l|}{Reserved} \\
\hline  32 & 0x020 & \multicolumn{3}{|l|}{Vcomp0} & \multicolumn{1}{|>{\columncolor[rgb]{0.82,0.82,0.82}}r|}{0} &
                     \multicolumn{3}{|l|}{Vstim0} & \multicolumn{1}{|>{\columncolor[rgb]{0.82,0.82,0.82}}r|}{0} \\
\hline  36 & 0x024 & \multicolumn{3}{|l|}{Vcomp1} & \multicolumn{1}{|>{\columncolor[rgb]{0.82,0.82,0.82}}r|}{0} &
                     \multicolumn{3}{|l|}{Vstim1} & \multicolumn{1}{|>{\columncolor[rgb]{0.82,0.82,0.82}}r|}{0} \\
\hline  40 & 0x028 & \multicolumn{3}{|l|}{Vcomp2} & \multicolumn{1}{|>{\columncolor[rgb]{0.82,0.82,0.82}}r|}{0} &
                     \multicolumn{3}{|l|}{Vstim2} & \multicolumn{1}{|>{\columncolor[rgb]{0.82,0.82,0.82}}r|}{0} \\
\hline  44 & 0x02C & \multicolumn{3}{|l|}{Vcomp3} & \multicolumn{1}{|>{\columncolor[rgb]{0.82,0.82,0.82}}r|}{0} &
                     \multicolumn{3}{|l|}{Vstim3} & \multicolumn{1}{|>{\columncolor[rgb]{0.82,0.82,0.82}}r|}{0} \\
\hline  48 & 0x030 & \multicolumn{3}{|l|}{Vcomp4} & \multicolumn{1}{|>{\columncolor[rgb]{0.82,0.82,0.82}}r|}{0} &
                     \multicolumn{3}{|l|}{Vstim4} & \multicolumn{1}{|>{\columncolor[rgb]{0.82,0.82,0.82}}r|}{0} \\
\hline  52 & 0x034 & \multicolumn{3}{|l|}{Vcomp5} & \multicolumn{1}{|>{\columncolor[rgb]{0.82,0.82,0.82}}r|}{0} &
                     \multicolumn{3}{|l|}{Vstim5} & \multicolumn{1}{|>{\columncolor[rgb]{0.82,0.82,0.82}}r|}{0} \\
\hline  56 & 0x038 & \multicolumn{3}{|l|}{Vcomp6} & \multicolumn{1}{|>{\columncolor[rgb]{0.82,0.82,0.82}}r|}{0} &
                     \multicolumn{3}{|l|}{Vstim6} & \multicolumn{1}{|>{\columncolor[rgb]{0.82,0.82,0.82}}r|}{0} \\
\hline  60 & 0x03C & \multicolumn{3}{|l|}{Vcomp7} & \multicolumn{1}{|>{\columncolor[rgb]{0.82,0.82,0.82}}r|}{0} &
                     \multicolumn{3}{|l|}{Vstim7} & \multicolumn{1}{|>{\columncolor[rgb]{0.82,0.82,0.82}}r|}{0} \\
\hline  64 & 0x040 & \multicolumn{2}{|c|}{I/E} & \multicolumn{2}{|c|}{Scmd} & \multicolumn{2}{|c|}{Cmd1} & \multicolumn{2}{|c|}{Cmd0} \\
\hline  68 & 0x044 & \multicolumn{6}{|>{\columncolor[rgb]{0.82,0.82,0.82}}l|}{Reserved} & \multicolumn{2}{|c|}{Vlcmd} \\
\hline  72 & 0x048 & \multicolumn{8}{|>{\columncolor[rgb]{0.82,0.82,0.82}}l|}{Reserved} \\
\hline  76 & 0x04C & \multicolumn{8}{|>{\columncolor[rgb]{0.82,0.82,0.82}}l|}{Reserved} \\
\hline  80 & 0x050 & \multicolumn{8}{|l|}{Interruption Counter} \\
\hline  84 & 0x054 & \multicolumn{8}{|>{\columncolor[rgb]{0.82,0.82,0.82}}l|}{Reserved} \\
\hline  88 & 0x058 & \multicolumn{8}{|>{\columncolor[rgb]{0.82,0.82,0.82}}l|}{Reserved} \\
\hline  92 & 0x05C & \multicolumn{8}{|>{\columncolor[rgb]{0.82,0.82,0.82}}l|}{Reserved} \\
\hline
\end{tabular}}\\
% some notes of usage:
% \multicolumn must either begin a row or be placed immediately after an &.
% \rowcolor[rgb]{0.8,0.8,0.8}; \columncolor[rgb]{0.82,0.82,0.82}

% improvements of seconds generation ASICs (1st layer)
The analog hardware is in charge of computing in continuous and
real-time neuronal and synaptic conductances. To increase the number
of neurons possibly processed by analog neural elements, the second
generation ASIC was able to model a set of 4 neurons and synaptic
conductances (per ASIC chip). {\2g} ASIC includes one inhibitory
neuron and three excitatory ones; each of them receives inhibitory
and excitatory synaptic inputs. This 1/3 ratios of RS and FS cells is
close to the ratio encountered in sensory cortex. Including four
neural elements per chip allows us to build larger network. \\


%------------------------------------------------------------------------------------------

%%% Local Variables: 
%%% mode: latex
%%% TeX-master: "../thesis"
%%% End:
